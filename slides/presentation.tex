\documentclass[aspectratio=43,unicode,10pt]{beamer}
\usetheme{ttipresentation}

\usepackage{graphicx}
\usepackage{multicol}

\beamertemplatenavigationsymbolsempty

\newcommand{\itemtitle}[1]{\textbf{#1}\\}
\newcommand{\fire}[1]{\textcolor{red}{\textbf{#1}}}
\newcommand{\freeze}[1]{\textcolor{blue}{\textbf{#1}}}
\newcommand{\then}{\textcolor{ttiblue}{\textbf{⇒}}\hspace{1ex}}
\newcommand{\good}{\textcolor{orange}{\textbf{◎}}\hspace{1ex}}
\newcommand{\arrow}{\textcolor{ttiblue}{\textbf{→}}\hspace{1ex}}


\title[font2char2word2sent2doc]{
  The New Model for Sentiment Classification \\
  of Documents Exploiting Fonts}
\institute[CoIn Lab., TTI]{Computational Intelligence Laboratory, \\
                      Toyota Technological Institute}
\author{16423 Yota Toyama}
\date{\today}



\begin{document}

\begin{frame}
\titlepage
\end{frame}

\begin{frame}{Sentiment classification of documents}
  \begin{itemize}
    \item aa
  \end{itemize}
  \begin{figure}
    \includegraphics[width=0.7\linewidth]{fig/review.png}
  \end{figure}
\end{frame}

\begin{frame}{Proposed Method}
  \begin{figure}
    \includegraphics[width=0.6\linewidth]{fig/fcwsd.pdf}
  \end{figure}
\end{frame}

% \begin{frame}{文間・カテゴリ間の関係}{}
%   \begin{columns}[t]
%     \begin{column}{0.6\textwidth}
%       \begin{block}{文間の関係}
%         「とても良かった」の文が
%         \begin{itemize}
%           \item 食事に関する文の直後に存在 \\
%                 \then 食事\good
%           \item 部屋に関する文の直後に存在 \\
%                 \then 部屋\good
%         \end{itemize}
%         \begin{figure}
%           \includegraphics[width=0.6\linewidth]
%                           {fig/global_relations_among_sentences_v2.pdf}
%         \end{figure}
%       \end{block}
%     \end{column}
%     \begin{column}{0.4\textwidth}
%       \begin{block}{カテゴリ間の関係}
%         \begin{itemize}
%           \item 他のカテゴリ\good \\
%                 \then 「総合」カテゴリ\good
%         \end{itemize}
%         \begin{figure}
%           \includegraphics[width=\linewidth]
%                           {fig/relations_among_rating_categories.pdf}
%         \end{figure}
%       \end{block}
%     \end{column}
%   \end{columns}
% \end{frame}

% \begin{frame}{関連研究}{}
%   \begin{block}{隠れ状態を用いたホテルレビューのレーティング予測
%                 \footnote[frame]{
%     藤谷宣典ら,
%     隠れ状態を用いたホテルレビューのレーティング予測.
%     言語処理学会第21回年次大会, 2015.
%   }\\(従来手法)}
%     \begin{itemize}
%       \item 文毎のレーティングからレビュー全体のレーティングを予測
%       \item カテゴリ間の繋がりを\fire{手調整で変化}させて考慮
%     \end{itemize}
%     \begin{figure}
%       \vspace{-1em} % HACK
%       \includegraphics[width=0.5\linewidth]
%                       {fig/fujitani_miml_relations_among_rating_categories.pdf}
%       \vspace{-1em} % HACK
%     \end{figure}
%   \end{block}
%   \begin{block}{パラグラフベクトル
%                 \footnote[frame]{
%     Quoc Le et al.,
%     Distributed representations of sentences and documents.
%     ICML 2014, 2014.
%   }}
%     \begin{itemize}
%       \item 文や文書を実数ベクトルに変換する手法
%       \item \fire{レーティング予測において優れた性能}
%     \end{itemize}
%   \end{block}
%   %\begin{block}{ニューラルネットワーク}
%   %  \begin{itemize}
%   %    %\item 神経回路を模した機械学習手法
%   %    %\item 分類問題に適用可能
%   %    \item \fire{入力間・出力間の複雑な関係}を考慮
%   %  \end{itemize}
%   %\end{block}
% \end{frame}

% \begin{frame}{提案手法}{}
%   \begin{itemize}
%     \item 文書・文間及びカテゴリ間の関係を自動で考慮した\\レーティング予測
%     \item パラグラフベクトルと\fire{入出力間の複雑な関係を考慮}できる \\
%           ニューラルネットワークを利用
%   \end{itemize}
%   %\begin{enumerate}
%   %  \item パラグラフベクトルにより各レビューとその中の文のベクトルを生成
%   %  \item 文ベクトルをレビュー毎に圧縮
%   %  \item ニューラルネットワークによりレーティングを予測
%   %\end{enumerate}
%   \begin{figure}
%     \includegraphics[width=0.8\linewidth]
%                     {fig/model_with_detailed_processes.pdf}
%   \end{figure}
% \end{frame}

% \begin{frame}{実験}{}
%   \begin{block}{実験設定}
%     \begin{itemize}
%       \item 7カテゴリにおける0〜5点のレーティング予測の正答率を測定
%       \item データセット:楽天トラベルにおけるレビュー約330,000件
%     \end{itemize}
%   \end{block}
%   \begin{block}{結果}
%     \begin{columns}[onlytextwidth,t]
%       \begin{column}{0.6\linewidth}
%         \begin{itemize}
%           \item 提案手法が従来手法より\fire{高い正答率}を示した
%           %\item \fire{文の並び}が予測のために重要
%           %\item 文書ベクトルと文ベクトルを同時に用いることが有効
%         \end{itemize}
%       \end{column}
%       \begin{column}{0.4\linewidth}
%         \begin{table}
%           \centering
%           \begin{tabular}{l | r}
%             手法 & 正答率 {[}\%{]} \\
%             \hline
%             従来手法 & 48.32 \\
%             提案手法 & \fire{50.30} \\
%           \end{tabular}
%         \end{table}
%       \end{column}
%     \end{columns}
%   \end{block}
% \end{frame}

% \begin{frame}{まとめと今後の予定}{}
%   \begin{block}{まとめ}
%     \begin{itemize}
%       \item 文書・文間及びカテゴリ間の関係を考慮した \\
%             レーティング予測手法を提案
%       \item 従来手法より高い正答率
%     \end{itemize}
%   \end{block}
%   \begin{block}{今後の予定}
%     \begin{itemize}
%       \item \fire{文間や単語間、文字間等のより多様な関係}を考慮 \\
%             \arrow レビューの特徴の抽出と分類の\fire{モデルを統合}
%     \end{itemize}
%   \end{block}
% \end{frame}

\end{document}
